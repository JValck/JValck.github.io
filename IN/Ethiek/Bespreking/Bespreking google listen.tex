\documentclass[11pt]{article}
\usepackage[hidelinks]{hyperref}
\author{Jasper De Valck}
\begin{document}
\section*{Bespreking: Does Google listen in on your life? Should it?}
Het artikel start met enkele getuigenissen van mensen die geconfronteerd werden met een correcte AutoComplete van Google. De voorbeelden die worden aangehaald zijn niet bepaald wereldschokkend, maar het achterliggende probleem maakt me meer ongerust.Als zulke voorspellingen gedaan kunnen worden, is het mogelijk dat Google de audio opnames bijhoudt. Vragen die dan bij mezelf opkomen zijn: hoelang worden deze audio opnames bijgehouden, maar vooral wat wordt ermee gedaan?\\

In het artikel wordt ook aangemaand om zelf te controleren welke data google allemaal bijhoudt. Dat heb ik dan ook zelf gedaan\footnote{Dit is mogelijk via: \url{https://www.google.com/settings/dashboard}}. Toen botste ik op enkele nostalgische feiten. Mijn e-mailadres van Google gaat immers al een decennium mee. Sinds het aanmaken van mijn account op 15 juli 2007 (ik was toen 13 jaar) heb ik duidelijk veel ge\"{e}xperimenteerd met de Googleproducten. Bij het controleren van de data kwam ik oude sites tegen die ik heb gemaakt. Dit is grappig, maar terzelfdertijd ook erg beangstigend. Ik heb dan ook mijn sterke bedenkingen in welke mate de doorsnee gebruiker op de hoogte is van de opgeslagen data.\\

Mensen die een telefoon aankopen die op Android draait, zijn verplicht om een Googleaccount aan te maken indien ze apps willen installeren vanaf Google Play. Sommigen hebben bij het aankopen van hun eerste Android telefoon nog geen Googleaccount en maken het aan. Uiteraard is Google zo slim om de privacy instellingen standaard zo open mogelijk in te stellen. Zo is bijvoorbeeld de Web- en app-activiteit (die verantwoordelijk is voor het opslagen en analyseren van zoekqueries) ingeschakeld.\\

Ikzelf heb al enkele jaren mijn privacy instellingen verbeterd. Alle aanpasbare privacy instellingen zijn intussen zo aangepast dat Google zo weinig mogelijk data van mij mag opslagen. Het artikel haalt ook aan dat Google hier transparant in wil zijn. Jammer genoeg stel ik iets anders vast op mijn activiteiten pagina. Zo weet Google perfect op welke advertenties ik heb geklikt\footnote{Dit constateerde ik via \url{https://myactivity.google.com/item}} (al dan niet per ongeluk). Ik vraag mij dan ook af welke gegevens Google nog van mij heeft verzameld en voor mezelf afschermt.
\end{document}