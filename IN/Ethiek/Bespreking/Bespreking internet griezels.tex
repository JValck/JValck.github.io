\documentclass[11pt]{article}
\usepackage[hidelinks]{hyperref}
\author{Jasper De Valck}
\begin{document}
\section*{Bespreking: Het internet van de griezels}
Steeds meer merk ik op Facebook dat vooral sensationele koppen veel aandacht krijgen. Wat me dan vaak opvalt is de oppervlakkigheid van de artikels die erachter schuilgaan.
Ik merk dat veel mensen die artikels (het fake nieuws) als waarheid nemen. Jammer.\\

Als toekomstige leraar voel ik mij dan ook verplicht om dit onderwerp aan te kaarten in toekomstige lessen. Ook bij leeftijdsgenoten merk ik dat ze zich vaak laten verleiden door sensationele koppen. Ik verwacht bijgevolg dat dit ook een probleem is bij de leerlingen uit het secundair.\\

Het artikel kaart - terecht - aan dat we ons beperken tot een bron van informatie. Die halen we dan vooral van Facebook en Google, aangezien we daar het meeste van de tijd doorbrengen. Belangrijk om op te merken is dat deze bedrijven ons zo goed mogelijk proberen te kennen en hierdoor enkel nieuwsbronnen aanrijken die aansluiten bij onze voorkeuren. Uiteindelijk vormt dit een spiraal en dan is het fake nieuws geboren. Ter verduidelijking: een nieuwsagentschap schrijft een artikel met kop "Trump wordt president" en koopt vervolgens een advertentie om dit artikel te promoten op Facebook.
Een gebruiker merkt het artikel op en laat zich verleiden om het artikel te lezen. Facebook leert hieruit dat zijn gebruiker ge\"{i}nteresseerd is in het presidentschap van Trump en zoekt in zijn advertentiedatabank naar andere artikels die over het presidentschap gaan. Merk op dat de gebruiker hierdoor steeds meer eenzijdige informatie krijgt.\\

Dit is dus een mogelijk gevaar van de hooggeprezen artifici\"{e}le intelligentie. Hierin schuilt ook een mogelijk gevaar. Wat als dit leidt tot het negatief afschilderen van bepaalde bevolkingsgroepen? Het zou niet de eerste keer zijn dat er een groot conflict ontstaat door slechte informatie.
\end{document}